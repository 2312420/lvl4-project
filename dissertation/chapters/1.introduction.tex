\chapter{Introduction}

% reset page numbering. Don't remove this!
\pagenumbering{arabic} 


\section{Motivation}
In 2019 alone, 60.359 trillion dollars’ \citep{website:worldbank} worth of stock was traded across the globe. In a matter of minutes, a company’s value can come crashing down with devastating effect or, adversely, soar up, making substantial many a profits for many.

It is the job of financial traders and analysts to analyse current trends and information to anticipate rises and falls in stock value in order to remain competitive and successful. One effective source of information is naturally the news. If the Wall street journal or BBC were to release a piece on a company failing to make a profit, it is reasonable to assume that many would see this piece and avoid investing in the company or cash out altogether, having a negative impact on its stock price. Thus through looking at news stories on a given company and how positive these stories are, we can begin to line up current sentiment and stock price to predict future values of the company.

However, although the millions of articles published and consumed every day provide an almost endless pool of data to use, it also makes it hard to sift through and gather this information to find the best investment opportunities. Therefore, it is the motivation of this project to develop a system that can collect and analyze articles to award a \textit{Hot} or \textit{Not} verdict for companies.

\section{Objectives}
The Objective of this project is to create a tool to aide financial traders in accessing  attractiveness of companies, based on the current sentiment gathered from news stories and articles. This assessment will be used to display an attractiveness score for a given company along with a prediction of its future stock price.

As mentioned, the system will first have to gather news articles from various reputable sources. It will then have to split these articles into manageable sentences, assess which companies are being discussed and ascertain the sentiment of the discussion. Using this information, along with historical stock data predictions for a company’s performance, a future stock price can begin to be created. Predictions and company attractiveness will be displayed through the development of a webapp, which should allow the user to find attractive or unattractive companies with ease, or to search for a specific one.

The effectiveness of this system will then be evaluated based on the performance of the individual system components, the accuracy of future stock price predictions and the corresponding attractiveness scores.


\section{Outline}

\textbf{Chapter \ref{Background}: Background} Looks into the already existing stock analytical webapps along with an academic review of the various fields the project covers. 
    

\textbf{Chapter \ref{Analysis}: Analysis / Requirements} Details the requirements and goals of the system along with how these requirements were gathered
    
\textbf{Chapter \ref{Design}: Design} Outlines the systems architecture and various components, and the various design choices made in development of the system

\textbf{Chapter \ref{Implementation}: Implementation} Describes the steps and challenges undertaken to implement the system
    
\textbf{Chapter \ref{Evaluation}: Evaluation} Shows how the key components of the system were evaluated and the improvements that were and the findings from these evaluations.  
    
\textbf{Chapter \ref{Conclusion}: Conclusion} Provides a summary and conclusion to the overall project along with the potential future work of the project. 
    
